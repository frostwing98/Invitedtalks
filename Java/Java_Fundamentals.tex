\documentclass[aspectratio=169,utf8]{beamer}
\usepackage[BoldFont,SlantFont]{xeCJK}
\usepackage{latexsym,amssymb,amsmath,amsbsy,amsopn,amstext,xcolor,multicol}
% \usepackage{graphicx,wrapfig,fancybox}
% \usepackage{pgf,pgfarrows,pgfnodes,pgfautomata,pgfheaps,pgfshade}
\usepackage{thubeamer}
% \usepackage{fontspec}
% \usepackage{tikz}
\usepackage{listings}
\usepackage{ulem}


\setsansfont{DejaVu Sans}

%\setframeofframes{of} % 1/10 --> curPage of totPage

\begin{document}

\title{Java Fundamentals 1:\\Intellij idea介绍与java基本介绍}
\author{\href{mailto:161250179@smail.nju.edu.cn}{frostwing98}}
\institute{SWI-NJU}
\frame{
  \titlepage


}
\section*{Table Of Contents}
\frame[allowframebreaks] {
  \frametitle{\secname}
    % \begin{multicols}{2}
    \tableofcontents
  % \end{multicols}

}

\AtBeginSubsection[] {
\frame<handout:0>[allowframebreaks] {
\frametitle{Outline}
\begin{multicols}{2}

\tableofcontents[current,currentsubsection]
\end{multicols}

  }
}
\section{First things first:开发环境搭建}
\frame{
如果你已经有idea和jdk可以跳过此部分。
}
\frame{
\textbf{工欲善其事,必先利其器}

\textbf{IDE}
java开发的ide主要有两种,eclipse与intellij idea。

使用Intellij Idea的\textbf{五}个理由
\begin{itemize}
  \item 多样的插件支持
  \item 多种语言支持
  \item 有爱的自动补全
  \item Jetbrains全家桶
  \item 学生免费!免费!
\end{itemize}

当然你可以试试用命令行编译什么的,一定比用记事本写c程序要恶心的多了
}
\frame{
\textbf{下载IDE}
\begin{itemize}
  \item 下载IDE很简单。到Jetbrains官网:https://www.jetbrains.com/idea/ 下载安装就行。
  \item 值得一提的是注册一个学生帐号可以获得免费的ultimate版本的使用权。ultimate版本提供了很多框架的功能,之后的之后可能会用到。
\end{itemize}

因为IDE仅仅只是工具而已,所以在哪里学习并不重要。因此我们不会重点介绍idea的功能,具体的教程可以自行参考。


}
\subsection{jdk}
\frame{
\textbf{安装jdk}
\begin{itemize}
  \item 安装完intellij之后的第二步就是下载jdk:https://www.oracle.com/technetwork/java/javase/downloads/jdk8-downloads-2133151.html
  \item 用的比较多的是java8,当然版本新一点也可以。需要根据你windows的版本下载x64(64位)/x86(32位)。下载后解压。别忘了设置环境变量JAVAHOME和PATH。
\end{itemize}

可以参考 https://jingyan.baidu.com/article/6dad5075d1dc40a123e36ea3.html 这个教程。


}
\section{第一个hello world}
\frame{
\textbf{新建项目}

\begin{itemize}
  \item 打开idea,新建项目-java,应该就能看到project SDK了。如果没有,需要手动制定jdk的bin文件夹。
  \item 在左侧的src文件夹右击,新 java class,我们就叫它demo好了。
\end{itemize}


\end{document}
