\documentclass[utf8]{beamer}
\usepackage[BoldFont,SlantFont]{xeCJK}
\usepackage{latexsym,amssymb,amsmath,amsbsy,amsopn,amstext,xcolor,multicol}
% \usepackage{graphicx,wrapfig,fancybox}
% \usepackage{pgf,pgfarrows,pgfnodes,pgfautomata,pgfheaps,pgfshade}
\usepackage{thubeamer}
% \usepackage{fontspec}
% \usepackage{tikz}
\usepackage{listings}
\usepackage{ulem}
\usepackage{graphicx}
\usepackage{tikz}

\setsansfont{DejaVu Sans}

%\setframeofframes{of} % 1/10 --> curPage of totPage

\begin{document}

\title{When JAVA going SMART:\\Android JNI Source Code Analysis}
\author{\href{mailto:161250179@smail.nju.edu.cn}{frostwing98}}
\institute{NJU-SWI}
\frame{
  \titlepage
}
\section*{Table Of Contents}
\frame[allowframebreaks] {
  \frametitle{\secname}
    % \begin{multicols}{2}
    \tableofcontents
  % \end{multicols}

}

\AtBeginSubsection[] {
\frame<handout:0>[allowframebreaks] {
\frametitle{Outline}
\begin{multicols}{2}

\tableofcontents[current,currentsubsection]
\end{multicols}

  }
}
\section{JNI Intro}
\subsection{What is JNI}
\frame{
\begin{figure}[ht]
\includegraphics[width=0.45\textwidth]{/home/allen/Pictures/cppfather.jpg}\hfill \pause
\includegraphics[width=0.45\textwidth]{/home/allen/Pictures/java.jpg}\hfill \pause
\caption{cpp \textbf{♂} java}
\end{figure}
}
\frame{
\frametitle{What is JNI}
\textbf{Java Native Interface}
\begin{itemize}
  \item A foreign function interface programming framework
  \item Enables Java code to call/be called by native applications
  \item Or libraries in C/C++
\end{itemize}
}
\subsection{JNI Everywhere}

\frame{
\frametitle{Applications of JNI}
\begin{figure}[ht]
\includegraphics[width=0.4\textwidth]{/home/allen/Pictures/5.png}\hfill \pause
\includegraphics[width=0.3\textwidth]{/home/allen/Pictures/ttpt1.png}\hfill \pause
\includegraphics[width=0.3\textwidth]{/home/allen/Pictures/6.png}
\caption{Android apps with DL frameworks}
\label{fig4}
\end{figure}
}
\subsection{JNI At First Glance}
\frame{

\frametitle{JNI:An example}


}
\section{JNI Analysis}
\subsection{JVM Init}
\frame{
\frametitle{From Init to Zygote}
\begin{itemize}
  \item After boot, Android starts \textbf{Init} process
  \item \textbf{Init} then calls \textbf{Zygote}
  \item \textbf{Zygote} is initiated.
  \href{https://android.googlesource.com/platform/frameworks/base/+/refs/heads/master/cmds/app_process/app_main.cpp}{\beamergotobutton{main->runtime.start()}}
  \item After \textbf{Init}, call \textbf{Zygote} which initiates Java VM and register jni methods.
\href{https://android.googlesource.com/platform/frameworks/base/+/refs/heads/master/core/jni/AndroidRuntime.cpp}{\beamergotobutton{AndroidRuntime::start()->startReg()->register\_jni\_procs()}}
\end{itemize}

}
\frame{
\frametitle{With Zygote into Runtime}
\textbf{Init I}
\begin{itemize}
  \item When calling register\_jni\_procs, register\_android\_os\_MessageQueue is called.
  \href{https://android.googlesource.com/platform/frameworks/base/+/refs/heads/master/core/jni/android_os_MessageQueue.cpp}{\beamergotobutton{register\_jni\_procs()->register\_android\_os\_MessageQueue()}}
  \item Then messageQueue registers the method.\href{https://android.googlesource.com/platform/frameworks/base/+/refs/heads/master/core/jni/core_jni_helpers.h}{\beamergotobutton{register\_android\_os\_MessageQueue()->RegisterMethodsOrDie()}}
  \item Each jni methods calls RegisterMethodsOrDie().
\end{itemize}
\textbf{Init II}
\begin{itemize}
  \item Runtime has several init methods, calling jniRegisterNativeMethods in JNIHelper.\href{https://android.googlesource.com/platform/libnativehelper/+/refs/heads/master/JNIHelp.cpp}{\beamergotobutton{Runtime->jniRegisterNativeMethods}}
  \item The JNIEnv let JNINativeInterface to process:\href{https://android.googlesource.com/platform/libnativehelper/+/refs/heads/master/include_jni/jni.h}{\beamergotobutton{\_JNIEnv struct}}
\end{itemize}
}
\subsection{Into JNIEnv}

\frame{
\frametitle{Into JNIEnv}
\begin{itemize}
  \item JNINativeInterface has function pointer to RegisterNative:\href{https://android.googlesource.com/platform/art/+/refs/heads/master/runtime/jni/jni_internal.cc}{\beamergotobutton{RegisterNative}}
  \item Then invoke RegisterNative let ArtMethod to register:\href{https://android.googlesource.com/platform/art/+/refs/heads/master/runtime/art_method.cc}{\beamergotobutton{JNINativeInterface->RegisterNative}}
\end{itemize}
}
\frame{
\frametitle{Summary}
\begin{itemize}
  \item On boot the system inits init process, then zygote
  \item Zygote is the father of all APP processes,  during the init the VM boots and framework JNI registered
  \item Zygote invoke register\_xxx method to register each module
  \item Each module invokes RegisterMethodsOrDie to register JNINativeMtehod array
  \item Eventually it goes into the VM to update each ARTMethod entry\_point\_from\_jni\_ by SetNativePointer
\end{itemize}
}
\subsection{JNI Load}
\frame{
\frametitle{Through loadlibrary}
\begin{itemize}
  \item System \href{https://android.googlesource.com/platform/libcore/+/refs/heads/master/ojluni/src/main/java/java/lang/System.java}{\beamergotobutton{System}}
  \item Runtime \href{https://android.googlesource.com/platform/libcore/+/refs/heads/master/ojluni/src/main/java/java/lang/Runtime.java }{\beamergotobutton{Runtime}}
  \item Then into jvm \href{https://android.googlesource.com/platform/art/+/refs/heads/master/openjdkjvm/OpenjdkJvm.cc}{\beamergotobutton{Runtime\_nativeLoad->JVM\_NativeLoad}}
  \item At last into JNI, \href{https://android.googlesource.com/platform/art/+/refs/heads/master/runtime/jni/java_vm_ext.cc}{\beamergotobutton{JVM\_NativeLoad->LoadNativeLibrary--->Java\_OnLoad}}
\end{itemize}
}
\frame{
\frametitle{Summary}
\begin{itemize}
  \item System.loadLibrary()
  \item LoadNativeLibrary
  \item JNI\_OnLoad
\end{itemize}
}
\end{document}
