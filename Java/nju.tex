\documentclass[utf8]{beamer}
\usepackage[BoldFont,SlantFont]{xeCJK}
\usepackage{latexsym,amssymb,amsmath,amsbsy,amsopn,amstext,xcolor,multicol}
\usepackage{graphicx,wrapfig,fancybox}
\usepackage{pgf,pgfarrows,pgfnodes,pgfautomata,pgfheaps,pgfshade}
\usepackage{njubeamer}
\usepackage{fontspec}
\usepackage{tikz}
\usetikzlibrary{shapes.geometric}


\tikzstyle{startstop} = [rectangle, rounded corners, minimum width = 2cm, minimum height=1cm,text centered, draw = black ,fill = red!40]
\tikzstyle{io} = [trapezium, trapezium left angle=70, trapezium right angle=110, minimum width=2cm, minimum height=1cm, text centered, draw=black,fill=green!30]
\tikzstyle{process} = [rectangle, minimum width=3cm, minimum height=1cm, text centered, draw=black,fill = yellow!50]
\tikzstyle{decision} = [diamond, aspect = 3, text centered, draw=black, fill = blue!40]
\tikzstyle{arrow} = [->,>=stealth]

\setsansfont{DejaVu Sans}

%\setframeofframes{of} % 1/10 --> curPage of totPage

\begin{document}

\title{GeekMark Crowdsourcing}
\author{frostwing98}
\institute{Nanjing University}
%\date{}

\frame{
\titlepage
\begin{figure}[htpb]
  \begin{center}
  \includegraphics[width=0.2\linewidth]{download.png}
  \end{center}
\end{figure}
}

  \section*{Table Of Contents}
  \frame {
    \frametitle{\secname}
    \tableofcontents
  }


  \AtBeginSubsection[] {
  \frame<handout:0> {
  \frametitle{Outline}
  \tableofcontents[current,currentsubsection]
    }
  }
  \section{什么是好的产品}
  \subsection{第一印象}
  \begin{figure}[]
    \begin{center}
    \includegraphics[width=1\linewidth]{screenshot.png}
    \end{center}
  \end{figure}

  \subsection{什么是好的产品?}
  \frame
  {
  "UFO"
    \frametitle{\secname~ }
    \begin{block}{UI Intereaction}
    \end{block}
    \begin{block}{Functionality}
    \end{block}
    \begin{block}{Optimization}
    \end{block}
    }
  \section{软件特性}
    \subsection{功能性}
      \subsubsection{举例:发布任务过程}

        \frame{

        让我们想象一下你是发布者……
        \begin{itemize}[<+-| alert@+>]
          \item
          发布任务... XD
          \item
          N分钟以后……
          \item
          !信息填错了!我怎么管不住我的手呢! 0.0
          \item
          删掉任务! :)
          \item
          重新发布.
        \end{itemize}
        }
        \frame{
        现在让我们想象一下你是一个工人...
          \begin{itemize}[<+-| alert@+>]
          \item
         打开网站... :)
          \item
         有新任务!! XD
          \item
         Done! 坐等评分. :)
          \item
         什么鬼!任务没了!垃圾平台,还我分数! QAQ
          \end{itemize}
        }
        \frame{
          \begin{itemize}[<+-| alert@+>]
          \item
         管理员:
          \item
         这什么数据啊,有mark没任务,搞什么啊?
          \item
         垃圾平台,我要跳槽!
          \end{itemize}
        }
        \frame{
        谁该背锅?
          \begin{itemize}[<+-| alert@+>]
          \item
            发起者!他管不住自己的手,就应该为自己的错误买单!
            \begin{enumerate}
            \item ——如果惩罚发起者,发起者不愿发起任务。
          \end{enumerate}
          \item
            工人背锅!就少给点分数意思一下保个底就行了……
            \begin{enumerate}
            \item
            ——工人:???
            \item
            ——工人流失。
          \end{enumerate}
          \item
            平台背锅!给工人钱!
            \begin{enumerate}
            \item——平台破产
            \item——全剧终。
            \end{enumerate}
          \item
            emmmmm……众人无语ing
          \end{itemize}
        }
        \frame{
        \frametitle{n次开会(争吵)之后……}
        }
        \frame{
        \frametitle{n次开会(争吵)之后……}

        \begin{block}{发布任务}
        \end{block}
        \begin{block}{取消任务}
        \end{block}
        \begin{block}{提前结束}
        \end{block}
        \begin{block}{延长任务}
        \end{block}
        }
        \subsubsection{举例:部署过程}
        \frame{
        现在让我们想象一下你是一个管理员……
        \begin{itemize}[<+-| alert@+>]
          \item ——部署文档怎么那么长啊……
          \item ——部署文档怎么看不懂啊……
          \item ——部署文档怎么有错啊……
        \end{itemize}
        }
        \frame{
        \includegraphics[width=0.8\linewidth]{1.png}
        \includegraphics[width=0.8\linewidth]{2.png}
        }
        \frame{
        \includegraphics[width=0.8\linewidth]{3.png}

        }
        \frame{
        \includegraphics[width=0.8\linewidth]{4.png}
        }
  \subsection{性能}
  \subsubsection{图片加载优化}

  \frame{
    %\frametitle{\subsecname~Goods}
	朝着更快进发……
    \begin{enumerate}[<+-| alert@+>]
    \item
	  图片压缩
    \includegraphics[width=0.4\linewidth]{screen.jpg}
  \end{enumerate}
  }
\frame{
\begin{enumerate}[<+-| alert@+>]
  \setcounter{enumi}{1}
  \item
  阿里云 OSS 对象存储服务
  \includegraphics[width=1\linewidth]{OSS.png}
\end{enumerate}
}
\frame{
\begin{enumerate}[<+-| alert@+>]
  \setcounter{enumi}{2}
  \item
  分布式标注信息存储
  \item
  分布式分数计算
  \begin{enumerate}
  \item——悲壮的尝试……
  \includegraphics[width=0.8\linewidth]{RIP.png}
  \end{enumerate}
\end{enumerate}

}
  \section{项目概述}
  \subsection{代码架构}
    \frame{
    \scalebox{0.75}{
      \begin{tikzpicture}[node distance=2cm]
      %定义流程图具体形状
      \node[process](front) at (5,7.5) {展示层};
      \node[process](Controller) at (5,6) {控制器};
      \node[process](BL) at (5,4.5) {BL层};
      \node[process](Repository) at (5,3) {bean仓库};
      \node[decision](OSS) at (12,4.5) {阿里云OSS};
      \node[process](Beans) at (0,1.5) {beans};
      \node[decision](DataBase) at (5,0) {数据库};
      %\node[process](Beans) at (0,5)
      \draw [arrow] (front) -- (Controller);
      \draw [arrow] (Controller) -- (BL);
      \draw [arrow] (BL) -- (Repository);
      \draw [arrow] (BL) -- (OSS);
      %\draw (Repository) -- node [below] {自动装配} (Beans);
      \draw [arrow] (Repository) -- (Beans);
      \draw [arrow] (Beans) -- (DataBase);
      \end{tikzpicture}
      }
    }


    \subsection{小组组成}
    \frame{
      \begin{tikzpicture}[node distance=2cm]
      \node[process](front) at (0,7.5) {张寅-前端};
      \node[startstop,right of = front, node distance=60mm](desc1){靠谱担当!};
      \node[process](Controller) at (0,6) {虞圣呈-BL};
      \node[startstop,right of = Controller,node distance=60mm](desc2){骚思路担当!!};
      \node[process](BL) at (0,4.5) {石沁轩-后端、文档};
      \node[startstop,right of = BL, node distance=60mm](desc3){救急担当!};
      \node[process](Repository) at (0,3) {杨宇清-层间交互-疯狂debug};
      \node[startstop,right of = Repository, node distance=60mm](desc4){多灾多难的PM};

      \end{tikzpicture}
    }
    \subsection{一个简单的添加功能过程}
    \frame{
    \scalebox{0.6}{
      \begin{tikzpicture}[node distance=1cm]
        \node[startstop](start){接受挑战!};
        \node[io, below of = start, yshift = -1cm](in1){开会分(抢)锅};
        \node[process, below of = in1, yshift = -1cm](pro1){BL写完,催前端};
        \node[process, below of = pro1, yshift = -1cm](dec1){前端写完,催数据库};
        \node[process, below of = dec1, yshift = -1cm](dec2){PM写完,催数据库};

        \coordinate (point1) at (-3cm, -6cm);
        %连接具体形状
        \draw [arrow] (start) -- (in1);
        \draw [arrow] (in1) -- (pro1);
        \draw [arrow] (pro1) -- (dec1);
        \draw [arrow] (dec1) -- (dec2);
      \end{tikzpicture}
    }
    }
    \frame{
    \scalebox{0.6}{
      \begin{tikzpicture}[node distance=1cm]
        \node[startstop](start){接受挑战!};
        \node[io, below of = start, yshift = -1cm](in1){开会分(抢)锅};
        \node[process, below of = in1, yshift = -1cm](pro1){BL写完,催前端};
        \node[process, below of = pro1, yshift = -1cm](dec1){前端写完,催数据库};
        \node[process, below of = dec1, yshift = -1cm](dec2){PM写完,催数据库};
        \node[process, below of = dec2, yshift = -1cm](pro2){PM写数据库};


        \coordinate (point1) at (-3cm, -6cm);
        %连接具体形状
        \draw [arrow] (start) -- (in1);
        \draw [arrow] (in1) -- (pro1);
        \draw [arrow] (pro1) -- (dec1);
        \draw [arrow] (dec1) -- (dec2);
        \draw [arrow] (dec2) -- (pro2);
      \end{tikzpicture}
    }
    }
    \frame{
    \scalebox{0.6}{
      \begin{tikzpicture}[node distance=1cm]
        \node[startstop](start){接受挑战!};
        \node[io, below of = start, yshift = -1cm](in1){开会分(抢)锅};
        \node[process, below of = in1, yshift = -1cm](pro1){BL写完,催前端};
        \node[process, below of = pro1, yshift = -1cm](dec1){前端写完,催数据库};
        \node[process, below of = dec1, yshift = -1cm](dec2){PM写完,催数据库};
        \node[process, below of = dec2, yshift = -1cm](pro2){PM写数据库};

        \node[process, right of = start, yshift = -1cm, node distance=120mm](pro3){PM合成系统};
        \node[process, below of = pro3, yshift = -1cm](out1){BUG出现};
        \node[process, below of = out1, yshift = -1cm](debug){PM debug};
        \node[process, below of = debug, yshift = -1cm](debug1){PM 通知大家改bug};
        \node[decision, below of = debug1,yshift=-1cm](debug2){bug de完没有?};
        \node[startstop, below of = debug2, yshift = -1cm](stop){PM debug至死};
        \coordinate (point1) at (-3cm, -6cm);
        %连接具体形状
        \draw [arrow] (start) -- (in1);
        \draw [arrow] (in1) -- (pro1);
        \draw [arrow] (pro1) -- (dec1);
        \draw [arrow] (dec1) -- (dec2);
        \draw [arrow] (dec2) -- (pro2);
        \draw [arrow] (pro2.east) -- (pro3.west);
        \draw [arrow] (pro3) -- (out1);
        \draw [arrow] (out1) -- (debug);
        \draw [arrow] (debug) -- (debug1);
        \draw [arrow] (debug1) -- (debug2);
        \draw[->](debug2.east)|-node[right]{Yes}(out1);
        \draw [arrow] (debug2) -- (stop);
      \end{tikzpicture}
    }
    }
\section{总结}
\frame{
\begin{enumerate}[<+-| alert@+>]
  \item{
  启示1
  }
  \begin{block}{会用轮子很重要,用好轮子更重要}
  \end{block}
  \item{
  启示2
  }
  \begin{block}{一个项目的架构与开发成员密切相关}
  \end{block}
  \item{
  启示3
  }
  \begin{block}{后期合成时流的泪都是前期架构时脑子进的水}
  \end{block}

\end{enumerate}
}
\frame{
\begin{enumerate}[<+-| alert@+>]
  \setcounter{enumi}{3}
    \item{
    启示4
    }
    \begin{block}{DDL-Driven:永恒的驱动逻辑}
    \end{block}
    \item{
    启示5
    }
    \begin{block}{BUG是DE不完的,这辈子都DE不完的}
    \end{block}
    \item{
    启示6
    }
    \begin{block}{开心就好}
    \end{block}
  \end{enumerate}
}
\subsection*{Thanks}



\frame{
  %\frametitle{\subsecname}
  \begin{columns}
  \column{3cm}
  \column{4cm}
    Thank you!
  \column{3cm}
  \end{columns}
}

\end{document}


%\frame{
%  %\frametitle{\subsecname~ frame b}
%  \begin{itemize}[<+-| alert@+>]
%  \item
%  item a
%  \end{itemize}
%}
%\begin{figure}
%\includegraphics[height=10cm,width=12cm]{a3.eps}
%\caption{}
%\label{a3}
%\end{figure}
