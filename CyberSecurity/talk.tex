\documentclass[aspectratio=169,utf8]{beamer}
\usepackage[BoldFont,SlantFont]{xeCJK}
\usepackage{latexsym,amssymb,amsmath,amsbsy,amsopn,amstext,xcolor,multicol}
% \usepackage{graphicx,wrapfig,fancybox}
% \usepackage{pgf,pgfarrows,pgfnodes,pgfautomata,pgfheaps,pgfshade}
\usepackage{thubeamer}
% \usepackage{fontspec}
% \usepackage{tikz}
% \usepackage{listings}
\usepackage{ulem}


\setsansfont{DejaVu Sans}

%\setframeofframes{of} % 1/10 --> curPage of totPage

\begin{document}

\title{Information Security Advances\\with a Focus on AI \& Privacy Preserving}
\author{\href{mailto:161250179@smail.nju.edu.cn}{frostwing98}}
\institute{COSEC of Nanjing University}
\frame{
  \titlepage


}
\section*{Table Of Contents}
\frame[allowframebreaks] {
  \frametitle{\secname}
    \begin{multicols}{2}
    \tableofcontents
  \end{multicols}

}

\AtBeginSubsection[] {
\frame<handout:0>[allowframebreaks] {
\frametitle{Outline}
\begin{multicols}{2}

\tableofcontents[current,currentsubsection]
\end{multicols}

  }
}
\section{Information Security}
\frame{
\begin{itemize}
  \item \sout{Traditional}
  \begin{itemize}\pause
    \item \sout{Vulnurables}\pause
    \item \sout{Cybersecurity}\pause
    \item \sout{Cryptography}\pause
  \end{itemize}
  \item Cross-Disciplinary
  \begin{itemize}\pause
    \item Fuzzing\pause
    \item Privacy\pause
    \item \sout{Security-schema}
  \end{itemize}
\end{itemize}
}
\section{Selected Topic: Fuzzing}
\frame{
\textbf{AI fuzzing}

Fuzzing:

\begin{itemize}
  \item an automated software \textbf{testing} technique
  \item provides invalid, unexpected, or random data as inputs to a computer program
  \item monitors exceptions such as crashes, failing built-in code assertions, or potential memory leaks
\end{itemize}

Two main categories:
\begin{itemize}
  \item Fuzzing for AI
  \item AI for fuzzing
\end{itemize}
}
\subsection{Backgrounds}
\subsubsection{AI fuzzing}
\frame{
\textbf{Fuzzing for AI}

Target: AI Components

\begin{itemize}
  \item neuron coverage\footnotemark[1]
  \item layer coverage\footnotemark[2]
  \item formal security method\footnotemark[3]
\end{itemize}
\footnotetext[1]{\href{https://arxiv.org/pdf/1705.06640.pdf}{\tiny [Kexin Pei et al., 2017]DeepXplore: Automated Whitebox Testing of Deep Learning Systems}}
\footnotetext[2]{\href{https://arxiv.org/pdf/1803.07519.pdf}{\tiny [Lei Ma et al., 2018]DeepGauge: Multi-Granularity Testing Criteria for Deep Learning Systems}}
\footnotetext[3]{\href{https://arxiv.org/pdf/1804.10829.pdf}{\tiny [Shiqi Wang et al., 2018]Formal Security Analysis of Neural Networks using Symbolic Intervals}}


}
\frame{
\textbf{AI for fuzzing}

Method: AI

\begin{itemize}
  \item RNN-based(LSTM+AFL)\footnotemark[4]
  \item CNN-based(CNN+gradient descent)\footnotemark[5]
  \item RL(Q-Learning)\footnotemark[6]
\end{itemize}
\footnotetext[4]{\href{https://arxiv.org/pdf/1711.04596.pdf}{\tiny [Mohit Rajpal et al., 2017]Not all bytes are equal: Neural byte sieve for fuzzing}}
\footnotetext[5]{\href{https://arxiv.org/pdf/1705.06640.pdf}{\tiny [Dongdong She et al., 2017]NEUZZ: Efficient Fuzzing with NeuralProgram Smoothing}}
\footnotetext[6]{\href{https://arxiv.org/pdf/1801.04589.pdf}{\tiny [Konstantin Bottinger et al., 2018]Deep Reinforcement Fuzzing}}


}
\subsubsection{Testing}
\frame{
\textbf{Testing}

\begin{itemize}
  \item Blackbox, testing functions without peering into internal structures or workings
  \item Whitebox, testing  internal structures or workings of an application
  \item Greybox, tests improper structure-caused defects, if any
\end{itemize}
}
\subsubsection{Coverage}
\frame{
  \textbf{Coverage}

  \textbf{Software testing measurement} for describing the degree to which the source code of a program is executed

  \begin{itemize}
    \item Edge Coverage
    \item Function Coverage
    \item Statement Coverage

  \end{itemize}
}
\subsubsection{CNN,RNN}
\frame{
\textbf{CNN,RNN}

Short view:
\begin{itemize}
  \item MLP: Simplest DNN with fully-connected layers
  \item CNN: +Hypo:Space-correlation, everywhere in CV
  \item RNN: +Hypo:Time-correlation, usually used in Speech Analytics
\end{itemize}
}

\subsection{Related Works-Fuzzing for AI}
\subsubsection{DeepXPlore}
\frame{
\textbf{DeepXPlore}\footnotemark[1]

\begin{itemize}
  \item Neuron coverage: coverage of neurons with outputs exceeding preset thresholds
  \item Goal: Optimize neuron coverage
  \item How: Gradient Descending aiming to find maximal value
\end{itemize}

\footnotetext[1]{\href{https://arxiv.org/pdf/1705.06640.pdf}{\tiny [Kexin Pei et al., 2017]DeepXplore: Automated Whitebox Testing of Deep Learning Systems}}


}
\frame{
\includegraphics{pics/DX1.png}
\hfill
\includegraphics[width=0.5\textwidth]{pics/DX2.png}
}
\frame{
\includegraphics[height=1\textheight]{pics/DX3.png}

}
\subsubsection{DeepGauge}
\frame{
\textbf{DeepGauge}\footnotemark[2]

\begin{itemize}
  \item Neuron coverage is not enough:
  \begin{itemize}
    \item k-multisection Neuron Coverage
    \item Neuron Boundary Coverage \\(Corner Region Coverage)
    \item Strong Neuron Activation Coverage \\(Corner Case Coverage)
  \end{itemize}
  \item Layer coverage:
  \begin{itemize}
    \item Top-k Neuron Coverage
    \item Top-k Neuron Patterns
  \end{itemize}
\end{itemize}

\footnotetext[2]{\href{https://arxiv.org/pdf/1803.07519.pdf}{\tiny [Lei Ma et al., 2018]DeepGauge: Multi-Granularity Testing Criteria for Deep Learning Systems}}
}
\frame{
\includegraphics[height=1\textheight]{pics/DG1.png}
}

\frame{
\includegraphics[height=0.8\textheight]{pics/DG2.png}
}
\subsubsection{ReluVal}
\frame{
\textbf{ReluVal}\footnotemark[3]

\begin{itemize}
  \item Formal Security: Mathematically declared secure properties
  \item Goal: Achieve an exhaustive, high-performance analysis method
  \item How: Symbolic intervals and Interval analysis
\end{itemize}

\footnotetext[3]{\href{https://arxiv.org/pdf/1804.10829.pdf}{\tiny [Shiqi Wang et al., 2018]Formal Security Analysis of Neural Networks using Symbolic Intervals}}


}
\frame{
\begin{itemize}
  \item Existing adversarial testing models:
    \begin{itemize}
      \item No guarantee of non-existence of adversarial examples
      \item My conjecture:
      \begin{itemize}
        \item Tend to overestimate %only sure there exists, not sure others not.
        \item The example might not be applied to real life
      \end{itemize}
    \end{itemize}

  \item High overhead of SMT
    \begin{itemize}
      \item especially for non-linear, non-convex function
    \end{itemize}
\end{itemize}

}
\frame{
\textbf{Goal: }

\footnotesize A system for \textbf{formally} checking \textbf{security properties} of \textbf{Relu-based DNNs}


\begin{itemize}
  \item High efficiency: "200 times on average"
  \item High accuracy: "a variety of optimizations to improve accuracy"
\end{itemize}
}
\frame[label=Model]{
\begin{columns}[c]
  \column{10cm}
    \begin{itemize}
      \item Target system: ACAS Xu
      \item Security property: input-output-based \hyperlink{Appendix}{\beamergotobutton{To security property}}
      \item Attacker model: similar to adversarial examples:
    \end{itemize}
    \quad\includegraphics[width=.9\linewidth]{pics/model-1.png}
  \column{8cm}
    \includegraphics[width=.7\linewidth]{pics/model.png}
\end{columns}
% \begin{center}
% \includegraphics[width=.5\linewidth]{pics/model.png}
% \end{center}
}
\frame{
\begin{itemize}
\item Main method: interval analysis
\item $Optimization_{1}$: symbolic Interval
\item $Optimization_{2}$: iterative refinement:\\ \quad(existence of Lipschitz Consistency)
\end{itemize}
}
\frame{
\textbf{A Working Example: aiming to verify whether safe or not}
\begin{columns}[c]
  \quad
  \column{6cm}
    $$Distance:\ x,$$
    $$Approaching\ angle:\ y$$
    $$Safe\ property:\ x\in [4,6]\ y \in [1,5]$$
  \column{5cm}
    \includegraphics[width=.8\linewidth]{pics/example.png}
  \column{5cm}
    \includegraphics[width=.8\linewidth]{pics/examplea.png}
\end{columns}
}
\frame{
\textbf{Dependency error}
\begin{itemize}
  \item Naively computing output intervals in this way suffers from high errors as it computes extremely loose bounds.
  \item Only a highly conservative estimation of the output range, too wide to be useful for checking any safety property.
\end{itemize}
}
\frame{
\textbf{Symbolic Interval and Iterative Refinement}

\begin{itemize}
  \item Symbolic interval propagation
  \begin{itemize}
    \item explicitly represent the intermediate computations of each neuron in terms of the symbolic intervals that encode the interdependency of the inputs to minimize overestimation
  \end{itemize}
  \item Iterative refinement
    \begin{itemize}
      \item The dependency error for Lipschitz continuous functions decreases as the width of intervals decreases
      \item Therefore, we can bisect the input interval by evenly dividing the interval into the union of two consecutive sub-intervals and reduce the overestimation
    \end{itemize}
\end{itemize}
}
\frame{
\textbf{Symbolic Interval and Iterative Refinement}
\begin{columns}[c]
  \quad
  \column{7cm}
    \begin{itemize}
      \item \footnotesize Symbolic interval propagation\\ \quad (algebraic operand preservation)
      \item Iterative refinement\\ \quad (even interval division)
    \end{itemize}
  \column{4.5cm}
    \includegraphics[width=4.8cm,height=3.6cm]{pics/exampleb.png}
  \column{4.5cm}
    \includegraphics[width=4.0cm,height=3.6cm]{pics/examplec.png}
\end{columns}
}
\frame{
\begin{center}
  \includegraphics[width=.6\linewidth]{pics/method.png}
\end{center}
}
\frame{
\begin{center}
  \includegraphics[width=.8\linewidth]{pics/ev1.png}
\end{center}
}
\frame{
\begin{center}
  \includegraphics[width=.8\linewidth]{pics/ev2.png}
\end{center}
}
% traditional ai fuzzing
\subsection{Related Works-AI fuzzing}

\subsubsection{NEUZZ}
\frame{
\textbf{NEUZZ}\footnotemark[4]

\begin{itemize}
  \item Goal: Find lightweight solutions with efficiency and reasonable accuracy for fuzzing
  \item How: use CNN to train with inputs: compute edge coverage
  \item How: Gradient decent twith sign(increase/decrease by 1)
\end{itemize}
\footnotetext[4]{\href{https://arxiv.org/pdf/1705.06640.pdf}{\tiny [Dongdong She et al., 2017]NEUZZ: Efficient Fuzzing with NeuralProgram Smoothing}}

}
\subsubsection{AFL+LSTM}
\frame{
\textbf{Neural byte sieve}\footnotemark[5]

\begin{itemize}
  \item Goal: Use Machine Learning to learn guiding strategy based on input history and code coverage

\end{itemize}
\footnotetext[5]{\href{https://arxiv.org/pdf/1711.04596.pdf}{\tiny [Mohit Rajpal et al., 2017]Not all bytes are equal: Neural byte sieve for fuzzing}}

}
\subsubsection{DRL}
\frame{
\textbf{Deep Reinforce Learning}\footnotemark[6]
\begin{itemize}
  \item Formalize fuzzing procedure into reinforcement learning
\end{itemize}
\footnotetext[6]{\href{https://arxiv.org/pdf/1801.04589.pdf}{\tiny [Konstantin Bottinger et al., 2018]Deep Reinforcement Fuzzing}}

}
\section{Selected topic:Privacy(Superficial)}
\subsection{Related Work}
\subsubsection{Data Poisoning}
\frame{
\textbf{Data Poisoning}
\begin{itemize}
  \item Given a picture
  \item Calculate how much perturbation to add
  \item Goal: Make Classifier misclassify or do other jobs
\end{itemize}

\footnotetext[7]{\href{https://arxiv.org/pdf/1412.6572.pdf }{\tiny [Ian J. Goodfellow et al., 2014]Explaining and Harnessing Adversarial Examples}}
}
\frame{
\includegraphics[height=0.8\textheight]{pics/PO1.png}
\\
$$\boldsymbol{\eta}=\epsilon sign(\bigtriangledown_{x}\ J(\boldsymbol{\theta},x,y))$$
}
\frame{
\center
\includegraphics[height=0.8\textheight]{pics/gib.png}

}
\subsubsection{Sound:Voice-Over-IP}
\frame{
\textbf{Acoustic Eavesdropping}
\begin{itemize}
  \item Previous method: weak assumption, strong adversarial examples
  \item Scene: Known keyboard type, known user habit of typing(sounds)
  \item Goal: Learn the user typing and tell keystroke
\end{itemize}
\footnotetext[8]{\href{https://arxiv.org/pdf/1412.6572.pdf}{\tiny [A. Compagno et al., 2017]Don't Skype and Type! Acoustic Eavesdropping in Voice-Over-IP}}
}
\frame{
\includegraphics[height=0.8\textheight]{pics/ST1.png}
\hfill
\includegraphics[height=0.8\textheight]{pics/ST2.png}

}
\subsubsection{The Visual Microphone}
\frame{
\textbf{The Visual Microphone}
\begin{itemize}
  \item Target: Human speaking
  \item Gadget: crabchip/tissue/flower...+high frame rate camera
  \item Goal: Rebuild sound signal from vibration
\end{itemize}
\footnotetext[9]{\href{https://people.csail.mit.edu/mrub/papers/VisualMic_SIGGRAPH2014.pdf}{\tiny [Abe Davis et al., 2014]The Visual Microphone: Passive Recovery of Sound from Video}}

}
\frame{
\includegraphics[height=0.9\textheight]{pics/voip1.png}
\hfill
\includegraphics[height=0.9\textheight]{pics/voip2.png}
}
\frame{
\includegraphics[height=0.9\textheight]{pics/voip3.png}
\hfill
\includegraphics[height=0.9\textheight]{pics/voip4.png}
}
\frame{
\includegraphics[height=1\textheight]{pics/voip5.png}
}\frame{
\includegraphics[height=1\textheight]{pics/voip6.png}
}
\subsubsection{Sound:Dolphin Attack}
\frame{
\textbf{Dolphin Attack}
\begin{itemize}
  \item Target:Voice Assistant
  \item Method: Convert human voice command into higher frequencies
  \item Result: iPad、iPhone、MacBooks、Apple Watch、Amazon Echo, ThinkPad T440p
\end{itemize}
\footnotetext[10]{\href{https://arxiv.org/pdf/1708.09537.pdf}{\tiny [Guoming Zhang et al., 2017]Dolphin Attack: Inaudible Voice Commands}}
}

\frame{
\includegraphics[height=1\textheight]{pics/DA1.png}

}
\frame{
\includegraphics[height=1\textheight]{pics/DA2.png}

}
\frame{
\includegraphics[height=1\textheight]{pics/DA3.png}

}
\end{document}
