% \documentclass[utf8]{beamer}
\documentclass[aspectratio=169,utf8]{beamer}
\usepackage[BoldFont,SlantFont]{xeCJK}
\usepackage{latexsym,amssymb,amsmath,amsbsy,amsopn,amstext,xcolor,multicol}
\usepackage{graphicx,wrapfig,fancybox}
\usepackage{pgf,pgfarrows,pgfnodes,pgfautomata,pgfheaps,pgfshade}
\usepackage{thubeamer}
\usepackage{fontspec}
\usepackage{tikz}
\usetikzlibrary{shapes.geometric}


\tikzstyle{startstop} = [rectangle, rounded corners, minimum width = 2cm, minimum height=1cm,text centered, draw = black ,fill = red!40]
\tikzstyle{io} = [trapezium, trapezium left angle=70, trapezium right angle=110, minimum width=2cm, minimum height=1cm, text centered, draw=black,fill=green!30]
\tikzstyle{process} = [rectangle, minimum width=3cm, minimum height=1cm, text centered, draw=black,fill = yellow!50]
\tikzstyle{decision} = [diamond, aspect = 3, text centered, draw=black, fill = blue!40]
\tikzstyle{arrow} = [->,>=stealth]

\setsansfont{DejaVu Sans}

%\setframeofframes{of} % 1/10 --> curPage of totPage

% \renewcommand{\baselinestretch}{0.5} \normalsize

\begin{document}

\title{The Art of War: Offensive Techniques in Binary Analysis}
\author{Yan Shoshitaishvili et. al.}
\institute{杨宇清}
%\date{}

\frame{
\titlepage
}
\section*{Table Of Contents}
\frame {
  \frametitle{\secname}
  \begin{multicols}{2}
  \tableofcontents
\end{multicols}
  % \tableofcontents
}


% \AtBeginSubsection[] {
% \frame<handout:0> {
% \frametitle{Outline}
% \begin{multicols}{2}
% \tableofcontents[current,currentsubsection]
% \end{multicols}
%   }
% }
%
% \begin{itemize}
% \end{itemize}
\section{Intro}
\subsection{Motivation}
\frame{
\frametitle{Stubborn Binary Flaws}
\begin{itemize}
  \item Low-level language has few security gurantees
  \item Simple flaws remain stubborn(e.g. buffer overflow)
  \item New emerging vulnerabilities: IoT...
\end{itemize}
}
\subsection{Offensive Tragedy}
\frame{
\frametitle{Problems of existing researches}
\begin{itemize}
  \item SOTA methods ends up in prototypes
  \item Massive workload \& public unavailability: impractical to replicit
\end{itemize}
}
\subsection{Goal}
\frame{
\frametitle{Solve the first issue}
\begin{itemize}
  \item Reproduce
  \item Mitigate
\end{itemize}
}
\section{Background}
\subsection{Backgrounds}
\frame{
\frametitle{Automated Binary Analysis}
\begin{itemize}
  \item Replayability
  \item Semantic insight
\end{itemize}
}

\frame{
\frametitle{Static Discovery}
\begin{itemize}
  \item Controll flow recovery
  \begin{itemize}
    \item Computed:target of jump is determined by calculation
    \item Context-sensitive: Determined by the caller
    \item Object-sensitive: In OO languages: e.g. virtual table
  \end{itemize}
  \item Flow modeling: requires pre-existing knowledge of the vulnerability
  \item Data modeling: by applying a tight over-estimation
\end{itemize}
}
\frame{
\frametitle{Dynamic Discovery}
\begin{itemize}
  \item Concrete Execution: executing in a minimally instrumented environment
  \begin{itemize}
    \item Coverage-based fuzzing: AFL, modifying input
    \item Taint-based fuzzing, tries to know how the program processes the data
  \end{itemize}
  \item Symbolic Execution: using abstract symbols to execute
  \begin{itemize}
    \item Classical: not very practical, bugs shallow
    \item Symbolic-assisted fuzzing: combines the advantages of fuzzing and symbolic execution
    \item Under-constrained symbolic execution: low replayability.
  \end{itemize}
\end{itemize}
}
\frame{
\frametitle{Exploitation}
\begin{itemize}
  \item Crash reproduction
  \begin{itemize}
    \item Missing data
    \item Missing relationships
  \end{itemize}
  \item Exploit generation: NULL-pointer
  \item Exploit hardening: modern techniques(ASLR,etc.) makes it less practical
\end{itemize}
}
\section{Design: Analysis Engine}
\frame{
\textbf{Goal:}
\begin{itemize}
  \item Cross-architecture
  \item Cross-platform
  \item Various analysis paradigm sypport
  \item Usability
\end{itemize}

\textbf{Modules:}
\begin{itemize}
  \item Imtermediate representation
  \item Binary Loading
  \item State Representation/Modification
  \item Data Model
  \item Full-Program Analysis
\end{itemize}
}
\frame{
\frametitle{Imtermediate representation}
libVEX
}
\frame{
\frametitle{Binary Loading}
CLE
}
\frame{
\frametitle{State Representation/Modification}
State plugins:
\begin{itemize}
  \item Registers
  \item Symbolic memory
  \item Abstract memory
  \item POSIX
  \item Log
  \item Inspection
  \item Solver
  \item Architecture
\end{itemize}
}
\frame{
\frametitle{Data Model}
Claripy: abstracts all values into internal representation
\begin{itemize}
  \item FullFrontend
  \item CompositeFrontend
  \item LightFrontend
  \item ReplacementFrontend
  \item HybridFrontend
\end{itemize}
}
\frame{
\frametitle{Full Program Analysis}
A combine of all the methods above.
}
\section{Implementations}
\frame{
\frametitle{CFG Recovery}
\begin{itemize}
  \item Based on several assumptions
  \item Iterative CFG Generation
  \begin{itemize}
  \item Forced Execution
  \item Symbolic Execution
  \item Backward Slicing
  \end{itemize}
  \item CFG Fast
  \begin{itemize}
    \item Function identification
    \item Recursive disassembly
    \item Indirect jump resolution
  \end{itemize}
\end{itemize}
}
\frame{
\frametitle{Value Set Analysis}
Improments against previous approaches:
\begin{itemize}
  \item Creating a discrete set of strided-intervals
  \item Applying an algebraic solver to path predicates
  \item Adopting a signedness-agnostic domain
\end{itemize}
}
\frame{
\frametitle{Dynamic Symbolic Execution}
\begin{itemize}
  \item Uses Claripy to populate symbolic model
  \item Uses Path object(angr) to track action,path, etc.
  \item Aggregate paths to PathGroup
\end{itemize}
}
\frame{
\frametitle{Under-constrained Symbolic Execution}
Two changes:
\begin{itemize}
  \item Global memory under-constraining
  \item False positive filtering
\end{itemize}
}
\frame{
\frametitle{Symbolic-assisted Fuzzing}
\begin{itemize}
  \item AFL executes until there is no new states
  \item Invoke angr on unique paths for new transitions
\end{itemize}
}
\frame{
\frametitle{Crash Reproduction}
\begin{itemize}
  \item Search problem: an input bringing program from $s_{0}$ to $s$
\end{itemize}
}
\frame{
\frametitle{Exploit Generation}
\begin{itemize}
  \item Vulnerable States
  \item Instruction Pointer Overwrite Technique
  \item Exploiting CGC Binaries
\end{itemize}
}
\frame{
\frametitle{Exploit Hardening}
Steps:
\begin{itemize}
  \item Gadget discovery
  \item Gadget arrangement
  \item Payload generation
\end{itemize}
}
\end{document}
