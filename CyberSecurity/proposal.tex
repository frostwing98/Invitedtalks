\documentclass[utf8]{beamer}
\usepackage{xeCJK}
\usepackage{latexsym,amssymb,amsmath,amsbsy,amsopn,amstext,xcolor,multicol}
\usepackage{graphicx,wrapfig,fancybox}
\usepackage{thubeamer}
\usepackage{fontspec}
\usepackage{tikz}


%\documentclass[aspectratio=169,utf8]{beamer}


\setsansfont{DejaVu Sans}

%\setframeofframes{of} % 1/10 --> curPage of totPage

% \renewcommand{\baselinestretch}{0.5} \normalsize

\begin{document}


\title{AI Model Testing Fuzzing:\\Research Project Proposal}
\author{Yuqing Yang}
\institute{Nanjing University}
%\date{}

\frame{
\titlepage
}
% \section*{Table Of Contents}
% \frame {
%   \frametitle{\secname}
%   \begin{multicols}{2}
%   \tableofcontents
% \end{multicols}
%   % \tableofcontents
% }


\AtBeginSection[] {
\frame<handout:0> {
\frametitle{Outline}
% \begin{multicols}{2}
\tableofcontents[current,currentsubsection]
% \end{multicols}
  }
}
%
% \begin{itemize}
% \end{itemize}
\section{Backgrounds}
\subsection{AI fuzzing}
\frame{
\textbf{AI fuzzing}

Fuzzing:

\begin{itemize}
  \item an automated software \textbf{testing} technique
  \item provides invalid, unexpected, or random data as inputs to a computer program
  \item monitors exceptions such as crashes, failing built-in code assertions, or potential memory leaks
\end{itemize}

Two main categories:
\begin{itemize}
  \item Fuzzing for AI
  \item AI for fuzzing
\end{itemize}
}
\subsubsection{Fuzzing for AI}
\frame{
\textbf{Fuzzing for AI}

Target: AI Components

\begin{itemize}
  \item neuron coverage\footnotemark[1]
  \item layer coverage\footnotemark[2]
  \item formal security method\footnotemark[3]
\end{itemize}
\footnotetext[1]{\href{https://arxiv.org/pdf/1705.06640.pdf}{\tiny [Kexin Pei et al., 2017]DeepXplore: Automated Whitebox Testing of Deep Learning Systems}}
\footnotetext[2]{\href{https://arxiv.org/pdf/1803.07519.pdf}{\tiny [Lei Ma et al., 2018]DeepGauge: Multi-Granularity Testing Criteria for Deep Learning Systems}}
\footnotetext[3]{\href{https://arxiv.org/pdf/1804.10829.pdf}{\tiny [Shiqi Wang et al., 2018]Formal Security Analysis of Neural Networks using Symbolic Intervals}}


}
\subsubsection{AI for fuzzing}
\frame{
\textbf{AI for fuzzing}

Method: AI

\begin{itemize}
  \item RNN-based(LSTM+AFL)\footnotemark[4]
  \item CNN-based(CNN+gradient descent)\footnotemark[5]
  \item RL(Q-Learning)\footnotemark[6]
\end{itemize}
\footnotetext[4]{\href{https://arxiv.org/pdf/1711.04596.pdf}{\tiny [Mohit Rajpal et al., 2017]Not all bytes are equal: Neural byte sieve for fuzzing}}
\footnotetext[5]{\href{https://arxiv.org/pdf/1705.06640.pdf}{\tiny [Dongdong She et al., 2017]NEUZZ: Efficient Fuzzing with NeuralProgram Smoothing}}
\footnotetext[6]{\href{https://arxiv.org/pdf/1801.04589.pdf}{\tiny [Konstantin Bottinger et al., 2018]Deep Reinforcement Fuzzing}}


}
\subsection{Testing}
\frame{
\textbf{Testing}

\begin{itemize}
  \item Blackbox, testing functions without peering into internal structures or workings
  \item Whitebox, testing  internal structures or workings of an application
  \item Greybox, tests improper structure-caused defects, if any
\end{itemize}
}
\subsection{Coverage}
\frame{
  \textbf{Coverage}

  \textbf{Software testing measurement} for describing the degree to which the source code of a program is executed

  \begin{itemize}
    \item Edge Coverage
    \item Function Coverage
    \item Statement Coverage

  \end{itemize}
}
\subsection{CNN,RNN}
\frame{
\textbf{CNN,RNN}

Short view:
\begin{itemize}
  \item MLP: Simplest DNN with fully-connected layers
  \item CNN: +Hypo:Space-correlation, everywhere in CV
  \item RNN: +Hypo:Time-correlation, usually used in Speech Analytics
\end{itemize}
}

\section{Related Works}
\subsection{DeepXPlore}
\frame{
\textbf{DeepXPlore}\footnotemark[1]

\begin{itemize}
  \item Neuron coverage: coverage of neurons with outputs exceeding preset thresholds
  \item Goal: Optimize neuron coverage
  \item How: Gradient Descending aiming to find maximal value
\end{itemize}

\footnotetext[1]{\href{https://arxiv.org/pdf/1705.06640.pdf}{\tiny [Kexin Pei et al., 2017]DeepXplore: Automated Whitebox Testing of Deep Learning Systems}}


}
\subsection{DeepGauge}
\frame{
\textbf{DeepGauge}\footnotemark[2]

\begin{itemize}
  \item Neuron coverage is not enough:
  \begin{itemize}
    \item k-multisection Neuron Coverage
    \item Neuron Boundary Coverage \\(Corner Region Coverage)
    \item Strong Neuron Activation Coverage \\(Corner Case Coverage)
  \end{itemize}
  \item Layer coverage:
  \begin{itemize}
    \item Top-k Neuron Coverage
    \item Top-k Neuron Patterns
  \end{itemize}
\end{itemize}

\footnotetext[2]{\href{https://arxiv.org/pdf/1803.07519.pdf}{\tiny [Lei Ma et al., 2018]DeepGauge: Multi-Granularity Testing Criteria for Deep Learning Systems}}


}
\subsection{ReluVal}
\frame{
\textbf{ReluVal}\footnotemark[3]

\begin{itemize}
  \item Formal Security: Mathematically declared secure properties
  \item Goal: Achieve an exhaustive, high-performance analysis method
  \item How: Symbolic intervals and Interval analysis
\end{itemize}

\footnotetext[3]{\href{https://arxiv.org/pdf/1804.10829.pdf}{\tiny [Shiqi Wang et al., 2018]Formal Security Analysis of Neural Networks using Symbolic Intervals}}


}
\section{Research Goal}
\frame{
\begin{itemize}
  \item Goal: To explore new efficient way of fuzzing for AI components
  \item Target: Exsisting AI components
  \item How: Explore by adopting, analysing, optimizing exsisting fuzzing methods
  \item How: Optimize by combining suitable AI methods
\end{itemize}
}
\section{Refereences}
\frame{
\begin{itemize}
\item \href{https://arxiv.org/pdf/1705.06640.pdf}{\tiny [Kexin Pei et al., 2017]DeepXplore: Automated Whitebox Testing of Deep Learning Systems}
\item \href{https://arxiv.org/pdf/1803.07519.pdf}{\tiny [Lei Ma et al., 2018]DeepGauge: Multi-Granularity Testing Criteria for Deep Learning Systems}
\item \href{https://arxiv.org/pdf/1804.10829.pdf}{\tiny [Shiqi Wang et al., 2018]Formal Security Analysis of Neural Networks using Symbolic Intervals}
\item \href{https://arxiv.org/pdf/1711.04596.pdf}{\tiny [Mohit Rajpal et al., 2017]Not all bytes are equal: Neural byte sieve for fuzzing}
\item \href{https://arxiv.org/pdf/1705.06640.pdf}{\tiny [Dongdong She et al., 2017]NEUZZ: Efficient Fuzzing with NeuralProgram Smoothing}
\item \href{https://arxiv.org/pdf/1801.04589.pdf}{\tiny [Konstantin Bottinger et al., 2018]Deep Reinforcement Fuzzing}
\end{itemize}

}
\end{document}
